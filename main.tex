\documentclass[a4paper, fleqn, draft]{report}
\usepackage[T2A]{fontenc}
\usepackage[utf8]{inputenc}
\usepackage{amssymb}
\usepackage{amsmath}
\usepackage{amsthm}
\usepackage[left=2.5cm,right=2cm,top=2cm,bottom=2cm]{geometry}
\usepackage{xcolor}
\usepackage[english,russian]{babel}
\usepackage{mathtools}
\usepackage{caption}
\usepackage{subcaption}
\usepackage{tikz-cd}
\usetikzlibrary{babel}

\author{Itegulov D. S.}
\title{Category Theory course by Shtukenberg}
\date{Fall 2016 --- Winter 2017}

\newtheorem*{defn}{Определение} 
\newtheorem*{thm}{Теорема} 
\newtheorem*{stmt}{Утверждение}
\newtheorem*{exm}{Пример}
\newtheorem*{rem}{Замечание}
\newtheorem*{task}{Упражнение}

\newcommand{\isom}{\rightsquigarrow}
\newcommand{\oisom}{\cong}

\tikzset{
    labl/.style={anchor=south, rotate=90, inner sep=.5mm}
}
\tikzset{
    labrd/.style={anchor=south, rotate=-45}
}
\tikzset{
    labru/.style={anchor=south, rotate=45}
}
\tikzset{
    labr/.style={anchor=north, rotate=-90, inner sep=.5mm}
}

\begin{document}
\maketitle
\tableofcontents
\newpage
\chapter{Теория катерогий}
\section{Ввведение}
\subsection{Основные определения и свойства}
\begin{defn}
  Категорией называется такое
  $\mathcal{C} = \langle \mathcal{O}, hom, 1, \circ \rangle$, что:
  \begin{enumerate}
  \item Есть множество объектов $\mathcal{O}$: $(a, b, \ldots)$
  \item Для каждой пары объектов $(a, b)$ из $\mathcal{C}$, существует множество
    $hom(a, b)$, элементы которого называются морфизмами (стрелками) из $a$ в $b$:
    $(f \colon a \to b,\ g \colon b \to c, \ldots)$
  \item Для каждого объекта $a$ из $\mathcal{C}$ существует морфизм
    $1_a \colon a \to a$
  \item Для пары морфизмов $f \colon a \to b$ и $g \colon b \to c$
    определен морфизм $g \circ f \colon a \to c$, называемый композицией $f$ и $g$
  \end{enumerate}
  А также удовлетворяющее следующим аксиомам:
  \begin{itemize}
  \item Композиция ассоциативна, т.е. $f \circ (g \circ h) = (f \circ g) \circ h$
  \item Для любых морфизмов $f \colon a \to b, g \colon b \to c$ выполняется
    $f \circ 1_a = f$ и $1_a \circ g = g$ (Также называется законом тождества)
  \end{itemize}
\end{defn}

\begin{figure}[b]
  \centering
  \begin{tikzcd}[column sep=1in, row sep=1in]
    a \arrow[r, "f"] \arrow[dr, "g\circ f" labrd, pos=0.3, crossing over]
      \arrow[d, xshift=4pt, "(h \circ g) \circ f" labl, outer sep=8pt]
      \arrow[d, xshift=-4pt, "h \circ (g \circ f)" labl, outer sep=-9pt]
  & b \arrow [d, "g"]
      \arrow [ld, "h\circ g" labru, pos=0.3, swap, crossing over]
    \\
    d
  & c \arrow [ l , swap , " h " ]
  \end{tikzcd}

  \caption{Коммутативная диаграмма какой-то категории}
  \label{fig:cd1}
\end{figure}
\begin{exm}
  Несколько простых примеров категорий:
  \begin{enumerate}
    \item $Set$ -- это катерогия множеств \\
      Объекты в ней -- все множества \\
      Морфизмы -- отображения \\
      $1_a$ -- тождественное отображение \\
      $\circ$ -- обычная композиция отображений
    \item Категория $1$ \\
      $1_a\colon a \to a$ (нарисовать петлю)
    \item Категория $2$ \\
      две петли и $f\colon a \to b$
    \item Категория предпорядка \\
      Если есть множество объектов $\mathcal{O}$ и операция $\leq$ на них,
      обладающая свойствами рефлексивности и транзитивности, тогда это категория,
      объектами которой являются $\mathcal{O}$, стрелка между $a$ и $b$
      добавляется тогда и только тогда, когда $a \leq b$. Закон тождества
      выполняется в силу рефлексивности $\leq$, а ассоциативность $\circ$ в силу
      транзитивности $\leq$.
    \item Категория моноид \\
      Пусть есть моноид $M = \langle A, *, e \rangle$. Тогда в качестве
      единственного объекта возьмём $A$ ($\mathcal{O} = \{ A \}$),
      стрелками будут все элементы $A$ ($hom(A, A) = A$), $1_A = e$, а композиция
      стрелок будет эквивалентна $a * b$.
  \end{enumerate}
\end{exm}

\begin{figure}
\centering
\begin{subfigure}{.33\textwidth}
  \centering
  \begin{tikzcd}
    a \arrow[loop]
  \end{tikzcd}
  \caption{Категория 1}
  \label{fig:category-1}
\end{subfigure}%
\begin{subfigure}{.33\textwidth}
  \centering
  \begin{tikzcd}[column sep=1in]
    a \arrow[r, "f"] \arrow[loop]
  & b \arrow[loop]
  \end{tikzcd}
  \caption{Категория 2}
  \label{fig:category-2}
\end{subfigure}
\begin{subfigure}{.33\textwidth}
  \centering
  \begin{tikzcd}[column sep=1in]
    A \arrow[loop left, "e"] \arrow[loop above, "a"] \arrow[loop right, "b"]
    \arrow[loop below, "a * b"]
  \end{tikzcd}
  \caption{Категория моноид}
  \label{fig:category-2}
\end{subfigure}
\caption{Примеры категорий}
\label{fig:category-example}
\end{figure}

\begin{thm}
  В категории $\mathcal{C}$ для каждого объекта $x$ существует единственное $1_x$
\end{thm}
\begin{proof}
  Предположим есть две таких единицы $1_x$ и $I_x$.
  Тогда $1_x \circ I_x = 1_x$ (по первой половине закона тождества)
  и также $1_x \circ I_x = I_x$ (по второй половине закона тождества).
  Тогда по единственности композиции $1_x = I_x$
\end{proof}

\begin{defn}
  Дискретная категория -- такая категория, в которой единственные морфизмы --
  единичные стрелки.
\end{defn}
\begin{defn}
  Подкатегория -- $\mathcal{C} \subset \mathcal{D}$, если все
  $\mathcal{C}$-объекты это $\mathcal{D}$-объекты и все $\mathcal{C}$-морфизмы
  это $\mathcal{D}$-морфизмы.
\end{defn}
\begin{defn}
  Подкатегория называется полной, если для любых таких $\mathcal{C}$-объектов
  $a, b$, в $\mathcal{D}$ нет дополнительных морфизмов между $a$ и $b$.
\end{defn}
\begin{defn}
  Произведение категорий -- $\mathcal{C} \times \mathcal{D}$, если объектами
  являются пары $\langle a, b \rangle$, а морфизмы получаются из морфизмов
  $f\colon a \to c$, $g\colon b \to d$ следующим образом:
  $\langle f, g \rangle\colon \langle a, b \rangle \to \langle c,d \rangle$
\end{defn}

% Отсюда начинается какое-то невнятное гавно
$\mathcal{C}$ -- категория. Тогда посторим $\overrightarrow{\mathcal{C}}$: \\
$f\colon a \to b$ \\
$g\colon a \to b$ \\

Объекты $\overrightarrow{\mathcal{C}}$ -- $\langle h, k \rangle$ \\
$1_{f\colon a \to b} = \langle 1_a, 1_b \rangle$ \\

*Рисунок из трёх стрелок, показывающий композицию* \\
% Здесь невнятное гавно заканчивается

\begin{defn}
  Морфизм $f \colon a \to b$ называется мономорфизмом, если для любых морфизмов
  $g_1 \colon c \to a, g_2 \colon c \to a$, таких что $f \circ g_1 = f \circ g_2$
  следует $g_1 = g_2$.
\end{defn}

\begin{thm}
  Возьмём категорию $Set$: $f(g(x)) = f(h(x))$ влечёт $g(x)=h(x)$ | $f$ инъективно
\end{thm}
\begin{proof}
  $\Rightarrow$) Предположим $f$ не инъективно тогда найдутся такие $(u, v)$, что $f(u) = f(v)
  \Rightarrow g(x) = u h(x) = v противоречат$ \\
  $\Leftarrow$) Пусть для $x: g(x) \ne h(x)$ $f(g(x)) \ne f(h(x))$
\end{proof}

\begin{defn}
  Морфизм $f \colon a \to b$ называется эпиморфизмом, если для любых морфизмов
  $g_1 \colon b \to c, g_2 \colon b \to c$, таких что $g_1 \circ f = g_2 \circ f$
  следует $g_1 = g_2$
\end{defn}

\begin{defn}
  Морфизм $f \colon a \to b$ называется изоморфизмом, если найдётся
  $g \colon b \to a$, такая что $g \circ f = 1_a$ и $f \circ g = 1_b$.
  Изоморфная стрелка обозначается следующим образом:
  $f \colon a \rightsquigarrow b$
\end{defn}

\begin{stmt}
  $g$ из определения изоморфизма $f$ единственно
\end{stmt}
\begin{proof}
  Пусть существуют два таких морфизма $g$ и $g'$ из определения изоморфизма. Тогда:
  $g' = 1_a \circ g' = (g \circ f) \circ g' = g \circ (f \circ g') = g \circ 1_b = g$
\end{proof}

Эквивалентна ли изоморфность и эпиморфность с мономорфностью?
\begin{exm}
  Не эквивалентны. $\mathbb{N}$ -- пример категории (как моноид по $+$) с
  эпиморфными и мономорфными стрелками, но не с изоморфными стрелками.
\end{exm}

\begin{task}
  Пусть $f$ -- изоморфизм, тогда $f$ -- эпиморфизм и мономорфизм
\end{task}
\begin{proof}
  Докажем, что $f$ -- эпиморфизм. Мономорфизм доказывается аналогично.
  \begin{gather*}
    g_1 \circ f = g_2 \circ f \Rightarrow
    (g_1 \circ f) \circ f^{-1} = (g_2 \circ f) \circ f^{-1} \Rightarrow
    g_1 \circ (f \circ f^{-1}) = g_2 \circ (f \circ f^{-1}) \Rightarrow \\
    \Rightarrow g_1 \circ 1_b = g_2 \circ 1_b \Rightarrow
    g_1 = g_2
  \end{gather*}
\end{proof}

\begin{task}
  $1_x$ -- изоморфизм
\end{task}
\begin{proof}
  Докажем, что $I_x$ (переименованная $1_x$) удовлетворяет свойствам обратного к $1_x$:
  $I_x \circ 1_x = 1_x$, $1_x \circ I_x = 1_x$
\end{proof}

\begin{task}
  Пусть $f$ -- изоморфизм, тогда $f^{-1}$ -- изоморфизм
\end{task}
\begin{proof}
  Есть $g = f^{-1}$ из определения изоморфизма. Тогда для $g$ есть $f$, которое
  удовлетворяет свойствам из определения изоморфизма: $f \circ g = 1_b$ и
  $g \circ f = 1_a$
\end{proof}

\begin{task}
  Пусть $f \colon a \to b$ и $g \colon b \to c$ -- изоморфизмы, тогда $g \circ f$ -- изоморфизм, причём
  $(g \circ f)^{-1} = f^{-1} \circ g^{-1}$
\end{task}
\begin{proof}
  Докажем, что $f^{-1} \circ g^{-1}$ удовлетворяет свойствам обратного к
  $g \circ f$:
  \begin{gather*}
    (f^{-1} \circ g^{-1}) \circ (g \circ f) =
    ((f^{-1} \circ g^{-1}) \circ g) \circ f =
    (f^{-1} \circ (g^{-1} \circ g)) \circ f =
    (f^{-1} \circ 1_b) \circ f =
    f^{-1} \circ f = 1_a
  \end{gather*}
  Второе свойство доказывается аналогично
\end{proof}
\end{document}
\documentclass[a4paper, fleqn, draft]{report}
\usepackage[T2A]{fontenc}
\usepackage[utf8]{inputenc}
\usepackage{amssymb}
\usepackage{amsmath}
\usepackage{amsthm}
\usepackage[left=2.5cm,right=2cm,top=2cm,bottom=2cm]{geometry}
\usepackage{xcolor}
\usepackage[english,russian]{babel}
\usepackage{mathtools}
\usepackage{tikz-cd}
\usetikzlibrary{babel}

\author{Itegulov D. S.}
\title{Category Theory course by Shtukenberg}
\date{Fall 2016 --- Winter 2017}

\newtheorem*{defn}{Определение} 
\newtheorem*{thm}{Теорема} 
\newtheorem*{stmt}{Утверждение}

\begin{document}
\maketitle
\tableofcontents
\newpage
\chapter{Теория катерогий}
\section{Ввведение}
\subsection{Категория}
Баззворды: семантика Скотта, $D_{\inf}$
\begin{defn}
Категория $\mathcal{C}$ -- это:
\begin{enumerate}
\item Объекты: $(a, b, \ldots)$
\item Для каждой пары объектов заданы морфизмы (стрелки):
  $(f \colon a \to b,\ g \colon b \to c, \ldots)$
\item Для пары морфизмов $f \colon a \to b$ и $g \colon b \to c$
      определена композиция: $g \circ f\colon a \to c$ \\
      Композиция ассоциативна $f \circ (g \circ h) = (f \circ g) \circ h$
\item Должна существовать единица (для каждого объекта): \\
      $f \circ 1_a = f$ и $1_a \circ g = g$ (Закон тождества)
\end{enumerate}
\end{defn}
\tikzset{
    labl/.style={anchor=south, rotate=90, inner sep=.5mm}
}
\tikzset{
    labrd/.style={anchor=south, rotate=-45}
}
\tikzset{
    labru/.style={anchor=south, rotate=45}
}
\tikzset{
    labr/.style={anchor=north, rotate=-90, inner sep=.5mm}
}

\begin{figure}
  \centering
  \begin{tikzcd}[column sep=1in, row sep=1in]
    a \arrow[r, "f"] \arrow[dr, "g\circ f" labrd, pos=0.3, crossing over]
      \arrow[d, xshift=4pt, "(h \circ g) \circ f" labl, outer sep=8pt]
      \arrow[d, xshift=-4pt, "h \circ (g \circ f)" labl, outer sep=-9pt]
  & b \arrow [d, "g"]
      \arrow [ld, "h\circ g" labru, pos=0.3, swap, crossing over]
    \\
    d
  & c \arrow [ l , swap , " h " ]
  \end{tikzcd}

  \caption{Коммутативная диаграмма какой-то категории}
  \label{fig:cd1}
\end{figure}
Примеры:
\begin{enumerate}
  \item $Set$ -- катерогия множеств \\
        Объекты в ней -- множества \\
        Морфизмы -- отображения
  \item Категория $1$ -- $1_a\colon a \to a$ (нарисовать петлю) 
  \item Категория $2$ -- две петли и $f\colon a \to b$
  \item Предпорядок -- возьмём рефлексивность + транзитивность, тогда это категория
                       в которой между $a$ и $b$ есть не более одной стрелки
  \item Моноид $M = \langle A, *, e \rangle$ \\
        $a * b \Rightarrow a \circ b$ \\
        $e \Rightarrow 1_e$
\end{enumerate}



\begin{thm}
  Для каждого объекта $x$ существует единственное $1_x$
\end{thm}
\begin{proof}
  Предположим есть две таких единицы $1_x$ и $I_x$: \\
  Тогда $1_x \circ I_x = 1_x$ и также $1_x \circ I_x = I_x$ \\
  По единственности композиции $1_x = I_x$
\end{proof}

\subsection{Дискретная категория}
\begin{defn}
Дискретная категория -- такая категория, в которой единственные морфизмы --
единичные стрелки.
\end{defn}
\begin{defn}
Подкатегория -- $\mathcal{C} \subset \mathcal{D}$, если все $\mathcal{C}$-объекты это $\mathcal{D}$-объекты и все
$\mathcal{C}$-морфизмы это $\mathcal{D}$-морфизмы.
\end{defn}
\begin{defn}
Подкатегория называется полной, если для любых таких $\mathcal{C}$-объектов $a, b$, в $\mathcal{D}$ нет
дополнительных морфизмов между $a$ и $b$: $f\colon a \to b$.
\end{defn}
\begin{defn}
Произведение категорий $\mathcal{C} \times \mathcal{D}$ -- объектами являются пары $\langle a, b \rangle$,
а морфизмы получаются из морфизмов $f\colon a \to c$, $g\colon b \to d$ следующим образом:
$\langle f, g \rangle\colon \langle a, b \rangle \to \langle c,d \rangle$
\end{defn}

\subsection{Категория стрелок}
$\mathcal{C}$ -- категория. Тогда посторим $\overrightarrow{\mathcal{C}}$: \\
$f\colon a \to b$ \\
$g\colon a \to b$ \\

Объекты $\overrightarrow{\mathcal{C}}$ -- $\langle h, k \rangle$ \\
$1_{f\colon a \to b} = \langle 1_a, 1_b \rangle$ \\

*Рисунок из трёх стрелок, показывающий композицию* \\

$f\colon a \rightarrowtail b$ - мономорфная стрелка \\
Если для любых $g, h$ *квадратик <h f f g, a b c d>* влечёт $g = h$, то стрелка
мономорфна \\

Возьмём категорию $Set$: $f(g(x)) = f(h(x))$ влечёт $g(x)=h(x)$ | $f$ инъективно
\begin{proof}
  $\Rightarrow$) Предположим $f$ не инъективно тогда найдутся такие $(u, v)$, что $f(u) = f(v)
  \Rightarrow g(x) = u h(x) = v противоречат$ \\
  $\Leftarrow$) Пусть для $x: g(x) \ne h(x)$ $f(g(x)) \ne f(h(x))$
\end{proof}

Эпиморфные стрелки \\
$g \circ f = h \circ f$ влечёт $g = h$ \\
Обозначается как $f\colon a \twoheadrightarrow g$ \\
Аналог: Сюръективные функции

Изоморфная стрелка \\
$f\colon a \to b$ -- изоморфная стрелка, если найдётся $g\colon b \to a$, что
$g \circ f = 1_a$ и $f \circ g = 1_b$ \\
Обозначается как $f\colon a \rightsquigarrow b$ \\
\begin{stmt}
  $g$ единственно
\end{stmt}
\begin{proof}
  $g' = 1_a \circ g' = (g \circ f) \circ g' = g \circ (f \circ g') = g \circ 1_b = g$
\end{proof}

Пример категории в которой эпиморфность и мономорфность не влечёт изоморфность:
$\mathbb{N}$ \\
Упражнение: \\
\begin{enumerate}
  \item $f$ -- изострелка, то она эпистрелка и монострелка \\
  \item $1_x$ -- изострелка \\
  \item $f^{-1}$ -- изострелка \\
  \item $f$ и $g$ -- изострелки, то $f \circ g$ -- изострелка, причём
        $(f \circ g)^{-1} = g^{-1} \circ f^{-1}$ \\
\end{enumerate}
\end{document}
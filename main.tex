\documentclass[a4paper, fleqn, draft]{report}
\usepackage[T2A]{fontenc}
\usepackage[utf8]{inputenc}
\usepackage{amssymb}
\usepackage{amsmath}
\usepackage{amsthm}
\usepackage[left=2.5cm,right=2cm,top=2cm,bottom=2cm]{geometry}
\usepackage{xcolor}
\usepackage[english,russian]{babel}
\usepackage{mathtools}
\usepackage{caption}
\usepackage{subcaption}
\usepackage{tikz-cd}
\usetikzlibrary{babel}

\author{Itegulov D. S.}
\title{Category Theory course by Shtukenberg}
\date{Fall 2016 --- Winter 2017}

\newtheorem*{defn}{Определение} 
\newtheorem*{thm}{Теорема} 
\newtheorem*{stmt}{Утверждение}
\newtheorem*{exm}{Пример}
\newtheorem*{rem}{Замечание}
\newtheorem*{task}{Упражнение}

\newcommand{\isom}{\rightsquigarrow}
\newcommand{\oisom}{\cong}

\DeclareMathOperator{\dom}{dom}
\DeclareMathOperator{\cod}{cod}
\DeclareMathOperator{\pr}{pr}
\DeclareMathOperator{\inj}{in}

\tikzset{
    labl/.style={anchor=south, rotate=90, inner sep=.5mm}
}
\tikzset{
    labrd/.style={anchor=south, rotate=-45}
}
\tikzset{
    labru/.style={anchor=south, rotate=45}
}
\tikzset{
    labr/.style={anchor=north, rotate=-90, inner sep=.5mm}
}

\begin{document}
\maketitle
\tableofcontents
\newpage
\chapter{Теория катерогий}
\section{Ввведение}
\subsection{Основные определения и свойства}
\begin{defn}
  Категорией называется такое
  $\mathcal{C} = \langle \mathcal{O}, hom, 1, \circ \rangle$, что:
  \begin{enumerate}
  \item Есть множество объектов $\mathcal{O}$: $(a, b, \ldots)$
  \item Для каждой пары объектов $(a, b)$ из $\mathcal{C}$, существует множество
    $hom(a, b)$, элементы которого называются морфизмами (стрелками) из $a$ в $b$:
    $(f \colon a \to b,\ g \colon b \to c, \ldots)$
  \item Для каждого объекта $a$ из $\mathcal{C}$ существует морфизм
    $1_a \colon a \to a$
  \item Для пары морфизмов $f \colon a \to b$ и $g \colon b \to c$
    определен морфизм $g \circ f \colon a \to c$, называемый композицией $f$ и $g$
  \end{enumerate}
  А также удовлетворяющее следующим аксиомам:
  \begin{itemize}
  \item Композиция ассоциативна, т.е. $f \circ (g \circ h) = (f \circ g) \circ h$
  \item Для любых морфизмов $f \colon a \to b, g \colon b \to c$ выполняется
    $f \circ 1_a = f$ и $1_a \circ g = g$ (Также называется законом тождества)
  \end{itemize}
\end{defn}

\begin{figure}[b]
  \centering
  \begin{tikzcd}[column sep=1in, row sep=1in]
    a \arrow[r, "f"] \arrow[dr, "g\circ f" labrd, pos=0.3, crossing over]
      \arrow[d, xshift=4pt, "(h \circ g) \circ f" labl, outer sep=8pt]
      \arrow[d, xshift=-4pt, "h \circ (g \circ f)" labl, outer sep=-9pt]
  & b \arrow [d, "g"]
      \arrow [ld, "h\circ g" labru, pos=0.3, swap, crossing over]
    \\
    d
  & c \arrow [ l , swap , " h " ]
  \end{tikzcd}

  \caption{Коммутативная диаграмма какой-то категории}
  \label{fig:cd1}
\end{figure}
\begin{exm}
  Несколько простых примеров категорий:
  \begin{enumerate}
    \item $Set$ -- это катерогия множеств \\
      Объекты в ней -- все множества \\
      Морфизмы -- отображения \\
      $1_a$ -- тождественное отображение \\
      $\circ$ -- обычная композиция отображений
    \item Категория $1$ \\
      $1_a\colon a \to a$ (нарисовать петлю)
    \item Категория $2$ \\
      две петли и $f\colon a \to b$
    \item Категория предпорядка \\
      Если есть множество объектов $\mathcal{O}$ и операция $\leq$ на них,
      обладающая свойствами рефлексивности и транзитивности, тогда это категория,
      объектами которой являются $\mathcal{O}$, стрелка между $a$ и $b$
      добавляется тогда и только тогда, когда $a \leq b$. Закон тождества
      выполняется в силу рефлексивности $\leq$, а ассоциативность $\circ$ в силу
      транзитивности $\leq$.
    \item Категория моноид \\
      Пусть есть моноид $M = \langle A, *, e \rangle$. Тогда в качестве
      единственного объекта возьмём $A$ ($\mathcal{O} = \{ A \}$),
      стрелками будут все элементы $A$ ($hom(A, A) = A$), $1_A = e$, а композиция
      стрелок будет эквивалентна $a * b$.
  \end{enumerate}
\end{exm}

\begin{figure}
\centering
\begin{subfigure}{.33\textwidth}
  \centering
  \begin{tikzcd}
    a \arrow[loop]
  \end{tikzcd}
  \caption{Категория 1}
  \label{fig:category-1}
\end{subfigure}%
\begin{subfigure}{.33\textwidth}
  \centering
  \begin{tikzcd}[column sep=1in]
    a \arrow[r, "f"] \arrow[loop]
  & b \arrow[loop]
  \end{tikzcd}
  \caption{Категория 2}
  \label{fig:category-2}
\end{subfigure}
\begin{subfigure}{.33\textwidth}
  \centering
  \begin{tikzcd}[column sep=1in]
    A \arrow[loop left, "e"] \arrow[loop above, "a"] \arrow[loop right, "b"]
    \arrow[loop below, "a * b"]
  \end{tikzcd}
  \caption{Категория моноид}
  \label{fig:category-2}
\end{subfigure}
\caption{Примеры категорий}
\label{fig:category-example}
\end{figure}

\begin{thm}
  В категории $\mathcal{C}$ для каждого объекта $x$ существует единственное $1_x$
\end{thm}
\begin{proof}
  Предположим есть две таких единицы $1_x$ и $I_x$.
  Тогда $1_x \circ I_x = 1_x$ (по первой половине закона тождества)
  и также $1_x \circ I_x = I_x$ (по второй половине закона тождества).
  Тогда по единственности композиции $1_x = I_x$
\end{proof}

\begin{defn}
  Дискретная категория -- такая категория, в которой единственные морфизмы --
  единичные стрелки.
\end{defn}
\begin{defn}
  Подкатегория -- $\mathcal{C} \subset \mathcal{D}$, если все
  $\mathcal{C}$-объекты это $\mathcal{D}$-объекты и все $\mathcal{C}$-морфизмы
  это $\mathcal{D}$-морфизмы.
\end{defn}
\begin{defn}
  Подкатегория называется полной, если для любых таких $\mathcal{C}$-объектов
  $a, b$, в $\mathcal{D}$ нет дополнительных морфизмов между $a$ и $b$.
\end{defn}
\begin{defn}
  Произведение категорий -- $\mathcal{C} \times \mathcal{D}$, если объектами
  являются пары $\langle a, b \rangle$, а морфизмы получаются из морфизмов
  $f\colon a \to c$, $g\colon b \to d$ следующим образом:
  $\langle f, g \rangle\colon \langle a, b \rangle \to \langle c,d \rangle$
\end{defn}

% Отсюда начинается какое-то невнятное гавно
$\mathcal{C}$ -- категория. Тогда посторим $\overrightarrow{\mathcal{C}}$: \\
$f\colon a \to b$ \\
$g\colon a \to b$ \\

Объекты $\overrightarrow{\mathcal{C}}$ -- $\langle h, k \rangle$ \\
$1_{f\colon a \to b} = \langle 1_a, 1_b \rangle$ \\

*Рисунок из трёх стрелок, показывающий композицию* \\
% Здесь невнятное гавно заканчивается

\begin{defn}
  Морфизм $f \colon a \to b$ называется мономорфизмом, если для любых морфизмов
  $g_1 \colon c \to a, g_2 \colon c \to a$, таких что $f \circ g_1 = f \circ g_2$
  следует $g_1 = g_2$.
\end{defn}

\begin{thm}
  Возьмём категорию $Set$: $f(g(x)) = f(h(x))$ влечёт $g(x)=h(x)$ | $f$ инъективно
\end{thm}
\begin{proof}
  $\Rightarrow$) Предположим $f$ не инъективно тогда найдутся такие $(u, v)$, что $f(u) = f(v)
  \Rightarrow g(x) = u h(x) = v противоречат$ \\
  $\Leftarrow$) Пусть для $x: g(x) \ne h(x)$ $f(g(x)) \ne f(h(x))$
\end{proof}

\begin{defn}
  Морфизм $f \colon a \to b$ называется эпиморфизмом, если для любых морфизмов
  $g_1 \colon b \to c, g_2 \colon b \to c$, таких что $g_1 \circ f = g_2 \circ f$
  следует $g_1 = g_2$
\end{defn}

\begin{defn}
  Морфизм $f \colon a \to b$ называется изоморфизмом, если найдётся
  $g \colon b \to a$, такая что $g \circ f = 1_a$ и $f \circ g = 1_b$.
  Изоморфная стрелка обозначается следующим образом:
  $f \colon a \rightsquigarrow b$
\end{defn}

\begin{stmt}
  $g$ из определения изоморфизма $f$ единственно
\end{stmt}
\begin{proof}
  Пусть существуют два таких морфизма $g$ и $g'$ из определения изоморфизма. Тогда:
  $g' = 1_a \circ g' = (g \circ f) \circ g' = g \circ (f \circ g') = g \circ 1_b = g$
\end{proof}

Эквивалентна ли изоморфность и эпиморфность с мономорфностью?
\begin{exm}
  Не эквивалентны. $\mathbb{N}$ -- пример категории (как моноид по $+$) с
  эпиморфными и мономорфными стрелками, но не с изоморфными стрелками.
\end{exm}

\begin{task}
  Пусть $f$ -- изоморфизм, тогда $f$ -- эпиморфизм и мономорфизм
\end{task}
\begin{proof}
  Докажем, что $f$ -- эпиморфизм. Мономорфизм доказывается аналогично.
  \begin{gather*}
    g_1 \circ f = g_2 \circ f \Rightarrow
    (g_1 \circ f) \circ f^{-1} = (g_2 \circ f) \circ f^{-1} \Rightarrow
    g_1 \circ (f \circ f^{-1}) = g_2 \circ (f \circ f^{-1}) \Rightarrow \\
    \Rightarrow g_1 \circ 1_b = g_2 \circ 1_b \Rightarrow
    g_1 = g_2
  \end{gather*}
\end{proof}

\begin{task}
  $1_x$ -- изоморфизм
\end{task}
\begin{proof}
  Докажем, что $I_x$ (переименованная $1_x$) удовлетворяет свойствам обратного к $1_x$:
  $I_x \circ 1_x = 1_x$, $1_x \circ I_x = 1_x$
\end{proof}

\begin{task}
  Пусть $f$ -- изоморфизм, тогда $f^{-1}$ -- изоморфизм
\end{task}
\begin{proof}
  Есть $g = f^{-1}$ из определения изоморфизма. Тогда для $g$ есть $f$, которое
  удовлетворяет свойствам из определения изоморфизма: $f \circ g = 1_b$ и
  $g \circ f = 1_a$
\end{proof}

\begin{task}
  Пусть $f \colon a \to b$ и $g \colon b \to c$ -- изоморфизмы, тогда $g \circ f$ -- изоморфизм, причём
  $(g \circ f)^{-1} = f^{-1} \circ g^{-1}$
\end{task}
\begin{proof}
  Докажем, что $f^{-1} \circ g^{-1}$ удовлетворяет свойствам обратного к
  $g \circ f$:
  \begin{gather*}
    (f^{-1} \circ g^{-1}) \circ (g \circ f) =
    ((f^{-1} \circ g^{-1}) \circ g) \circ f =
    (f^{-1} \circ (g^{-1} \circ g)) \circ f =
    (f^{-1} \circ 1_b) \circ f =
    f^{-1} \circ f = 1_a
  \end{gather*}
  Второе свойство доказывается аналогично
\end{proof}
\section{Двойственность}
\subsection{Принцип двойственности}
\begin{defn}
  Пусть $\Sigma$ -- утверждение в терминах языка теории категорий.
  Тогда двойственное утверждение $\Sigma^{op}$ получается заменой
  $\cod \to \dom$ на $\dom \to \cod$ и $f \circ g$ на $g^{op} \circ f^{op}$
\end{defn}

\begin{defn}
  $\mathcal{C}^{op}$ -- двойственная категория к $\mathcal{C}$, если:
  \begin{enumerate}
    \item Все объекты $\mathcal{C}^{op}$ -- это $\mathcal{C}$-объекты
    \item $f \colon a \to b$ -- стрелка из $\mathcal{C} \iff f^{op}
      \colon b \to a$ -- стрелка в $\mathcal{C}^{op}$
    \item $f \circ g$ определена в $\mathcal{C}^{op}$ $\iff$ $g^{op} \circ
      f^{op}$ определена в $\mathcal{C}^{op}$
      $(f \circ g)^{op} = g^{op} \circ f^{op}$
  \end{enumerate}
\end{defn}

\begin{task}
  $\Sigma$ -- утверждение в $\mathcal{C}$, то $\Sigma^{op}$ -- утверждение в $\mathcal{C}^{op}$
\end{task}

\begin{task}
  $1$ -- единственная до изоморфизма
\end{task}

\subsection{Произведение}
\begin{defn}
  Пусть $a, b$ -- объекты $\mathcal{C}$. Тогда $a \times b$ -- произведение, если
  найдутся такие $\pr_a \colon a \times b \to a$ и $\pr_b \colon a \times b \to b$,
  что при любых $f \colon c \to a$ и $g \colon c \to b$ существует едиственное
  $\langle f, g \rangle \colon c \to a \times b$, что
  $\pr_a \circ \langle f, g \rangle = f$ и $\pr_b \circ \langle f, g \rangle = g$
\end{defn}

\begin{figure}[h]
  \centering
  \begin{tikzcd}[row sep=0.5in, column sep=1in]
    & c \arrow[ld, "f", swap] \arrow[rd, "g"] \arrow[d, dashed, "\langle f\text{,} g \rangle"]& \\
    a & a\times b \arrow[l, "\pr_a"] \arrow[r, "\pr_b", swap] & b
  \end{tikzcd}
  
  \caption{Произведение}
  \label{fig:multiplication}
\end{figure}

\begin{stmt}
  В категории $Set$ $A \times B$ -- декартово произведение множеств $A$ и $B$
\end{stmt}
\begin{proof}
  Пусть $C$ -- некоторое множество из $Set$ и есть некоторые $f \colon C \to A$ и
  $g \colon C \to B$. Тогда $\langle f, g \rangle (c) = \langle f(c), g(c) \rangle$
\end{proof}

\begin{stmt}
  $a \times b$ в категории $\mathcal{C}$ единственна до изоморфизма
\end{stmt}
\begin{proof}
  \begin{figure}[h]
    \centering
    \begin{tikzcd}[column sep=2cm]
      & c \arrow[ld, "f", swap] \arrow[rd, "g"] \arrow{d}[dashed]{\langle f, g \rangle}& \\
    a & a\times b \arrow[l, "\pr_a"] \arrow[r, "\pr_b", swap] \arrow{d}[dashed]{\langle \pr_a, \pr_b \rangle}& b \\
      & d \arrow[ul, "p"]\arrow[ur, "q", swap] &
    \end{tikzcd}
    \caption{К доказательству единственности произведения}
    \label{fig:multiplication-uniqueness}
  \end{figure}
  Мы хотим доказать, что существует такая $h \colon a \times b \rightsquigarrow d$
  Утверждается, что он $h = \langle (\pr_a, \pr_b) \rangle$, а обратное к нему $h^{-1}
  = \langle (p, q) \rangle$
  Из единственности петли понятно, что $h^{-1} \circ h = 1_{a \times b}$ и $h
  \circ h^{-1} = 1_d$
  Но почему она единственна? Пускай есть какое-то другое $\Pi(A, B)$ и отображение
  $\Pi_A \colon \Pi(A, B) \to A$ и $\Pi_B \colon \Pi(A, B) \to B$
\end{proof}

\begin{task}
  В категории предпорядка
  $a \times b$ удовлетворяет свойствам:
  \begin{enumerate}
  \item $a \times b \sqsubseteq a$, $a \times b \sqsubseteq b$
      $a \times b$ -- нижняя грань
    \item $c \sqsubseteq a$ и $c \sqsubseteq b$ влечёт $c \sqsubseteq a \times b$
  \end{enumerate}
  Т.е. предпорядок $c (x)$ -- нижняя полурешётка
\end{task}
\begin{proof}
   Пусть есть некоторое $c$ и $f \colon c \to a$, $g \colon c \to b$.
   Пусть $a \times b$ -- нижняя грань и натуральным образом от неё исходят
   $pr_a$ и $pr_b$ (доказательства того что нижняя грань меньше $a$ и $b$).
   Необходимо доказать существование $\langle f, g \rangle$.
   Из определения нижней грани существует такой морфизм. А т.к. в предпорядке
   между двумя объектами морфизм всегда единственнен получаем то что хотели.
 \end{proof}

\begin{task}
  $N$, что такое $(\times)$ в $N$, где оно определено
\end{task}

\begin{task}
  $\langle \pr_a, \pr_b \rangle = 1_{a \times b}$
\end{task}

\begin{task}
  $\langle f, g \rangle = \langle k, h \rangle$ влечёт $f = k$ и $g = h$
\end{task}

\begin{task}
  $\langle f \circ h, g \circ h \rangle = \langle f, g \rangle \circ h$
\end{task}

\begin{task}
  $A \rightsquigarrow A \times \{0\}$ в $Set$
\end{task}

\begin{task}
  $1$ - конечный объект, то $A \rightsquigarrow A \times 1$
\end{task}

\begin{task}
  Пусть $I \colon a \to 1$, тогда $\langle 1_a, I_a \rangle$ - изо
\end{task}

\begin{tikzcd}
  0 \arrow{r} \arrow[loop left]
& 0' \arrow{l}
\end{tikzcd}


\begin{tikzcd}
  & d \arrow[ld, "p", swap] \arrow[rd, "q"] \arrow{dd}[dashed]{S}& \\
  a &  & b \\
  & a\times b \arrow[ul, "p"] \arrow[ur, "q", swap] &
\end{tikzcd}

\subsection{Копроизведение (сумма)}
\begin{defn}
  Копроизведение -- объект, двойственный произведению. Другими словами
  пусть $a, b$ -- объекта $\mathcal{C}$. Тогда $a + b$ -- копроизведение, если
  найдутся такие $in_a \colon a + b \to a$ и $in_b \colon a + b \to b$, что при
  любых $f \colon c \to a$ и $g \colon c \to b$ существует единственная $[f, g]
  \colon a + b \to c$, что $[f, g] \circ in_a = f$ и $[f, g] \circ in_b = g$ 
\end{defn}
\begin{figure}[h]
  \centering
  \begin{tikzcd}[row sep=0.5in, column sep=1in]
    & w & \\
    a \arrow[ru, "f"] \arrow[r, "\inj_a"] &
    a + b \arrow[u, dashed, "[f \text{,} g \text{]}"] &
    b\arrow[lu, "g", swap] \arrow[l, "\pr_b", swap]
  \end{tikzcd}
  \caption{Копроизведение}
  \label{fig:coproduct}
\end{figure}
\begin{exm}[Копроизведение в $Set$]
  Копроизведением двух множеств $A$ и $B$ является их дизъюнктивное объединение:
  \begin{gather*}
    A + B = \{\, \langle {a, 0} \rangle \mid a \in A \,\}
      \cup \{\, \langle {b, 1} \rangle \mid b \in B \,\} \\
    \inj_A(x) = \langle {x, 0} \rangle \quad
    \inj_B(x) = \langle {x, 1} \rangle \\
  \end{gather*}
  Теперь для некоторых $f \colon A \to W$ и $g \colon B \to W$ определим $[f, g]$
  следующим образом:
  \begin{gather*}
    [f, g](p) = \begin{cases}
      f(x), & p = \langle {x, 0} \rangle \\
      g(x), & p = \langle {x, 1} \rangle
    \end{cases}
  \end{gather*}
\end{exm}
\begin{exm}[Копроизведение в категории предпорядка]
  \leavevmode
  \begin{enumerate}
    \item $a \sqsubseteq a + b$, $b \sqsubseteq a + b$
    \item Если $a \sqsubseteq c$ и $ b \sqsubseteq c$, то $ a + b \sqsubseteq c$
  \end{enumerate}
\end{exm}

\subsection{Уравнители}
\begin{defn}
  Пусть $f, g \colon a \rightrightarrows b$ -- пара морфизмов, тогда $i \colon e \to
  a$ называется уравнителем, если:
  \begin{enumerate}
    \item $f \circ i = g \circ i$
    \item Для любого $h \colon c \to a$ найдётся единственный морфизм
      $k \colon c \to e$ такой, что если $f \circ h = g \circ h$,
      то $i \circ k = h$
  \end{enumerate}
\end{defn}

\begin{thm}
  Уравнитель -- мономорфизм
\end{thm}
\begin{proof}
  Пусть $i \circ p = i \circ q$, но при этом $p \neq q$. Рассмотрим $f \circ (i
  \circ p) = (f \circ i) \circ p = (g \circ i) \circ p = g \circ (i \circ p)$.
  Получилось, что $f \circ (i \circ p) = g \circ (i \circ p)$. Тогда по определнию
  уравнителя есть единственное такое $k$, что $i \circ k = i \circ p$. Значит $k
  = p$ и аналогично доказывается $k = q$.
\end{proof}

\begin{thm}
  Эпиморфный уравнитель -- изоморфизм
\end{thm}
\begin{proof}
  По условию если $f \circ i = g \circ i$, то $f = g$. Тогда представим, что $c
  = a$ и $h = 1_a$. Получили $f \circ 1_a = g \circ 1_a = f$, а значит
  по определнию уравнителя есть единственное такое $k$, что $i \circ k = 1_a$.
  Тогда $i \circ k \circ i = 1_a \circ i = i = i \circ 1_b$. Получается, что $i
  \circ k \circ i = i \circ 1_b$ и по определению мономорфизма (а уравнитель это
  мономорфизм по прошлой теореме) получаем $k \circ i = 1_b$. Таким образом $k$
  -- обратный элемент для $i$. Поэтому $i$ -- изоморфизм.
\end{proof}

%%% Local Variables:
%%% mode: latex
%%% TeX-master: "main"
%%% End:

\end{document}